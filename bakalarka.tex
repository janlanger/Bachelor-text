%% History:
% Pavel Tvrdik (26.12.2004)
%  + initial version for PhD Report
%
% Daniel Sykora (27.01.2005)
%
% Michal Valenta (3.12.2008)
% rada zmen ve formatovani (diky M. Duškovi, J. Holubovi a J. Žďárkovi)
% sjednoceni zdrojoveho kodu pro anglickou, ceskou, bakalarskou a diplomovou praci

% One-page layout: (proof-)reading on display
%\documentclass[11pt,oneside,a4paper]{book}
% Two-page layout: final printing
 \documentclass[11pt,twoside,a4paper]{book}   
%=-=-=-=-=-=-=-=-=-=-=-=--=%
% The user of this template may find useful to have an alternative to these 
% officially suggested packages:
\usepackage[czech, english]{babel}
\usepackage[T1]{fontenc} % pouzije EC fonty 
% pripadne pisete-li cesky, pak lze zkusit take:
%\usepackage[OT1]{fontenc} 
\usepackage[utf8]{inputenc}
%=-=-=-=-=-=-=-=-=-=-=-=--=%
% In case of problems with PDF fonts, one may try to uncomment this line:
%\usepackage{lmodern}
%=-=-=-=-=-=-=-=-=-=-=-=--=%
%=-=-=-=-=-=-=-=-=-=-=-=--=%
% Depending on your particular TeX distribution and version of conversion tools 
% (dvips/dvipdf/ps2pdf), some (advanced | desperate) users may prefer to use 
% different settings.
% Please uncomment the following style and use your CSLaTeX (cslatex/pdfcslatex) 
% to process your work. Note however, this file is in UTF-8 and a conversion to 
% your native encoding may be required. Some settings below depend on babel 
% macros and should also be modified. See \selectlanguage \iflanguage.
%\usepackage{czech}  %%%%%
%\usepackage[T1]{czech} %%%%[IL2] [T1] [OT1]
%=-=-=-=-=-=-=-=-=-=-=-=--=%

%%%%%%%%%%%%%%%%%%%%%%%%%%%%%%%%%%%%%%%
% Styles required in your work follow %
%%%%%%%%%%%%%%%%%%%%%%%%%%%%%%%%%%%%%%%
\usepackage{graphicx}
\usepackage{indentfirst} %1. odstavec jako v cestine.

\usepackage{k336_thesis_macros} % specialni makra pro formatovani DP a BP
 % muzete si vytvorit i sva vlastni v souboru k336_thesis_macros.sty
 % najdete  radu jednoduchych definic, ktere zde ani nejsou pouzity
 % napriklad: 
 % \newcommand{\bfig}{\begin{figure}\begin{center}}
 % \newcommand{\efig}{\end{center}\end{figure}}
 % umoznuje pouzit prikaz \bfig namisto \begin{figure}\begin{center} atd.


%%%%%%%%%%%%%%%%%%%%%%%%%%%%%%%%%%%%%
% Zvolte jednu z moznosti 
% Choose one of the following options
%%%%%%%%%%%%%%%%%%%%%%%%%%%%%%%%%%%%%
%\newcommand\TypeOfWork{Diplomová práce} \typeout{Diplomova prace}
% \newcommand\TypeOfWork{Master's Thesis}   \typeout{Master's Thesis} 
 \newcommand\TypeOfWork{Bakalářská práce}  \typeout{Bakalarska prace}
% \newcommand\TypeOfWork{Bachelor's Project}  \typeout{Bachelor's Project}


%%%%%%%%%%%%%%%%%%%%%%%%%%%%%%%%%%%%%
% Zvolte jednu z moznosti 
% Choose one of the following options
%%%%%%%%%%%%%%%%%%%%%%%%%%%%%%%%%%%%%
% nabidky jsou z: http://www.fel.cvut.cz/cz/education/bk/prehled.html

%\newcommand\StudProgram{Elektrotechnika a informatika, dobíhající, Bakalářský}
%\newcommand\StudProgram{Elektrotechnika a informatika, dobíhající, Magisterský}
% \newcommand\StudProgram{Elektrotechnika a informatika, strukturovaný, Bakalářský}
% \newcommand\StudProgram{Elektrotechnika a informatika, strukturovaný, Navazující magisterský}
 \newcommand\StudProgram{Softwarové technologie a management, Bakalářský}
% English study:
% \newcommand\StudProgram{Electrical Engineering and Information Technology}  % bachelor programe
% \newcommand\StudProgram{Electrical Engineering and Information Technology}  %master program


%%%%%%%%%%%%%%%%%%%%%%%%%%%%%%%%%%%%%
% Zvolte jednu z moznosti 
% Choose one of the following options
%%%%%%%%%%%%%%%%%%%%%%%%%%%%%%%%%%%%%
% nabidky jsou z: http://www.fel.cvut.cz/cz/education/bk/prehled.html

%\newcommand\StudBranch{Výpočetní technika}   % pro program EaI bak. (dobihajici i strukt.)
%\newcommand\StudBranch{Výpočetní technika}   % pro prgoram EaI mag. (dobihajici i strukt.)
%\newcommand\StudBranch{Softwarové inženýrství}            %pro STM
\newcommand\StudBranch{Web a multimedia}                  % pro STM
%\newcommand\StudBranch{Computer Engineering}              % bachelor programe
%\newcommand\StudBranch{Computer Science and Engineering}  % master programe


%%%%%%%%%%%%%%%%%%%%%%%%%%%%%%%%%%%%%%%%%%%%
% Vyplnte nazev prace, autora a vedouciho
% Set up Work Title, Author and Supervisor
%%%%%%%%%%%%%%%%%%%%%%%%%%%%%%%%%%%%%%%%%%%%

\newcommand\WorkTitle{Studentova Berlička III - import dat z KOSu}
\newcommand\FirstandFamilyName{Jan Langer}
\newcommand\Supervisor{Ing. Jiří Chludil}


% Pouzijete-li pdflatex, tak je prijemne, kdyz bude mit vase prace
% funkcni odkazy i v pdf formatu
\usepackage[
pdftitle={\WorkTitle},
pdfauthor={\FirstandFamilyName},
bookmarks=true,
colorlinks=true,
breaklinks=true,
urlcolor=black,
citecolor=black,
linkcolor=black,
unicode=true,
]
{hyperref}



% Extension posted by Petr Dlouhy in order for better sources reference (\cite{} command) especially in Czech.
% April 2010
% See comment over \thebibliography command for details.

\usepackage[square, numbers]{natbib}             % sazba pouzite literatury
\usepackage{url}
%\DeclareUrlCommand\url{\def\UrlLeft{<}\def\UrlRight{>}\urlstyle{tt}}  %rm/sf/tt
%\renewcommand{\emph}[1]{\textsl{#1}}    % melo by byt kurziva nebo sklonene,
\let\oldUrl\url
\renewcommand\url[1]{<\texttt{\oldUrl{#1}}>}




\begin{document}

%%%%%%%%%%%%%%%%%%%%%%%%%%%%%%%%%%%%%
% Zvolte jednu z moznosti 
% Choose one of the following options
%%%%%%%%%%%%%%%%%%%%%%%%%%%%%%%%%%%%%
\selectlanguage{czech}
%\selectlanguage{english} 

% prikaz \typeout vypise vyse uvedena nastaveni v prikazovem okne
% pro pohodlne ladeni prace


\iflanguage{czech}{
	 \typeout{************************************************}
	 \typeout{Zvoleny jazyk: cestina}
	 \typeout{Typ prace: \TypeOfWork}
	 \typeout{Studijni program: \StudProgram}
	 \typeout{Obor: \StudBranch}
	 \typeout{Jmeno: \FirstandFamilyName}
	 \typeout{Nazev prace: \WorkTitle}
	 \typeout{Vedouci prace: \Supervisor}
	 \typeout{***************************************************}
	 \newcommand\Department{Katedra počítačů}
	 \newcommand\Faculty{Fakulta elektrotechnická}
	 \newcommand\University{České vysoké učení technické v Praze}
	 \newcommand\labelSupervisor{Vedoucí práce}
	 \newcommand\labelStudProgram{Studijní program}
	 \newcommand\labelStudBranch{Obor}
}{
	 \typeout{************************************************}
	 \typeout{Language: english}
	 \typeout{Type of Work: \TypeOfWork}
	 \typeout{Study Program: \StudProgram}
	 \typeout{Study Branch: \StudBranch}
	 \typeout{Author: \FirstandFamilyName}
	 \typeout{Title: \WorkTitle}
	 \typeout{Supervisor: \Supervisor}
	 \typeout{***************************************************}
	 \newcommand\Department{Department of Computer Science and Engineering}
	 \newcommand\Faculty{Faculty of Electrical Engineering}
	 \newcommand\University{Czech Technical University in Prague}
	 \newcommand\labelSupervisor{Supervisor}
	 \newcommand\labelStudProgram{Study Programme} 
	 \newcommand\labelStudBranch{Field of Study}
}




%%%%%%%%%%%%%%%%%%%%%%%%%%    Poznamky ke kompletaci prace
% Nasledujici pasaz uzavrenou v {} ve sve praci samozrejme 
% zakomentujte nebo odstrante. 
% Ve vysledne svazane praci bude nahrazena skutecnym 
% oficialnim zadanim vasi prace.
%{
%\pagenumbering{roman} \cleardoublepage \thispagestyle{empty}
%\chapter*{Na tomto místě bude oficiální zadání vaší práce}
%\begin{itemize}
%\item Toto zadání je podepsané děkanem a vedoucím katedry,
%\item musíte si ho vyzvednout na studijním oddělení Katedry počítačů na Karlově náměstí,
%\item v jedné odevzdané práci bude originál tohoto zadání (originál zůstává po obhajobě na katedře),
%\item ve druhé bude na stejném místě neověřená kopie tohoto dokumentu (tato se vám vrátí po obhajobě).
%\end{itemize}
%\newpage
%}

%%%%%%%%%%%%%%%%%%%%%%%%%%    Titulni stranka / Title page 

\coverpagestarts

%%%%%%%%%%%%%%%%%%%%%%%%%%%    Podekovani / Acknowledgements 

\acknowledgements
\noindent
Zde můžete napsat své poděkování, pokud chcete a máte komu děkovat.


%%%%%%%%%%%%%%%%%%%%%%%%%%%   Prohlaseni / Declaration 

\declaration{V~Praze dne 15.\,5.\,2008}
%\declaration{In Kořenovice nad Bečvárkou on May 15, 2008}


%%%%%%%%%%%%%%%%%%%%%%%%%%%%    Abstract 
 
\abstractpage

Translation of Czech abstract into English.

% Prace v cestine musi krome abstraktu v anglictine obsahovat i
% abstrakt v cestine.
\vglue60mm

\noindent{\Huge \textbf{Abstrakt}}
\vskip 2.75\baselineskip

\noindent
Abstrakt práce by měl velmi stručně vystihovat její podstatu. Tedy čím se práce zabývá a co je jejím výsledkem/přínosem.

\noindent
Očekávají se cca 1 -- 2 odstavce, maximálně půl stránky.

%%%%%%%%%%%%%%%%%%%%%%%%%%%%%%%%  Obsah / Table of Contents 

\tableofcontents


%%%%%%%%%%%%%%%%%%%%%%%%%%%%%%%  Seznam obrazku / List of Figures 

\listoffigures


%%%%%%%%%%%%%%%%%%%%%%%%%%%%%%%  Seznam tabulek / List of Tables

\listoftables


%**************************************************************

\mainbodystarts
% horizontalní mezera mezi dvema odstavci
%\parskip=5pt
%11.12.2008 parskip + tolerance
\normalfont
\parskip=0.2\baselineskip plus 0.2\baselineskip minus 0.1\baselineskip

% Odsazeni prvniho radku odstavce resi class book (neaplikuje se na prvni 
% odstavce kapitol, sekci, podsekci atd.) Viz usepackage{indentfirst}.
% Chcete-li selektivne zamezit odsazeni 1. radku nektereho odstavce,
% pouzijte prikaz \noindent.


%*****************************************************************************
\chapter{Úvod}

V této úvodní kapitole naleznete stručnou historii dvou informačních systémů, kterých se tato práce bezprostředně dotýká - Komponenty Studium a Studentovy berličky.

\section{KOS a jeho historie}
Následující kapitola obsahuje krátký pohled do historie KOSu. Ačkoli informace vychází z malého počtu zdrojů, které navíc obsahují pouze kusé informace, jde o jediné zdroje, které se mi podařilo najít.
\subsection{Historický vývoj}
Historie Komponenty Studium se začala psát před osmnácti lety, v akademickém roce 1992/93\cite{forum:historie-kos}. Tehdejší děkan Fakulty elektrotechnické prof. Hlavička ustanovil komisi, která měla za úkol vytvoření informačního systému pro podporu studijních agend fakulty. Ačkoli původní plán byl takový, že systém vytvoří vlastními silami profesionálové z Katedry počítačů, nakonec jej ale na základě specifikací od akademických pracovníků z řady kateder vytvořila kladenská softwarová firma Tril, spol s.r.o.\cite{forum:neocekavana}.

Systém byl (a stále je) postaven nad databázovým strojem Oracle, uživatelské rozhraní bylo vygenerováno prostřednictvím Oracle Forms verze 4, a bylo přístupné prostřednictvím znakového terminálu. Zavedení systému značně zjednodušilo správu studijních agend a dovolilo studentům samostatně řešit některé úkoly bez nutnosti navštívit Pedagogické oddělení. Nedlouho po vytvoření systému jej zavedla také Fakulta strojní a později i zbylé fakulty ČVUT.

KOS přístupný pouze přes znakový terminál fungoval téměř deset let, do roku 2002. Za tu dobu se ve velké míře rozšířili osobní počítače a vysokorychlostní internet i do domácností, a tak sílily hlasy po webovém rozhraní. Bylo vypsáno výběrové řízení na WWW rozhraní KOSu pro role Student a Ucitel. Zadání specifikovalo společné WWW rozhraní pro celé ČVUT, které bude komunikovat se všemi třemi (FEL, FS a VIC) instancemi  databáze. Zároveň mělo být řešení postaveno na existujících základech a umožnit koexistenci s terminálovým přístupem. 

Zakázku vyhrála firma Trask Solutions, a.s. a webový KOS byl v září 2002 uveden do provozu, zatím jen nad instancí spravovanou ve VIC. K tomu uvádí Ing. Halaška\cite{student:komentar-ke-kos} \textit{\uv{Urychlené uvedení vybraných WWW formulářů do provozu k prvnímu září 2002 bylo možné (z našeho hlediska) pouze za cenu některých bezpečnostních kompromisů, kterých se my na FEL bojíme.}} To se nakonec ukázalo jako správné rozhodnutí, rozhraní totiž obsahovalo množství závažných chyb, a tak se WWW KOS zpřístupnil pro studenty a učitele na FEL až v září 2003\cite{student:kos-na-web}.

\subsection{Současný stav KOS}
Píše se rok 2010 a KOS má za sebou 18 let své existence. Po akci \uv{Sjednocení instancí KOS} z listopadu 2008\cite{sik} již běží pouze v jedné instanci pod správou VIC pro celou univerzitu. Je téměř neuvěřitelné, že systém přežil až do dnešních dní, navíc jak uvádí Ing. Halaška\cite{student:komentar-ke-kos} \textit{\uv{aniž by bylo nutné cokoliv zásadně měnit}}. Za těch 18 let ale udělal vývoj v informatice pořádný skok kupředu, a KOS dnes v porovnání s informačními systémy jiných českých univerzit\footnote{Za zmínku stojí především IS/STAG Západočeské univerzity nebo IS Masarykovy univerzity v Brně, který obdržel v roce 2005 ocenění EUNIS Elite Award. Systém je formou outsourcingu nasazován i na jiných školách v ČR. } již neobstojí.

Velkým problémem současného stavu je neexistence aplikačního rozhraní. V době kdy je ve velké míře prosazován svobodný přístup k informacím, má ČVUT informační systém, který je proprietární, a z vnějšku pro aplikace kateder v podstatě nepřístupný.


\section{Studentova berlička}

Studentova berlička je prací, kterou vytvořil Jiří Hunka při jeho studiu na ČVUT FEL a věnoval se jí ve své bakalářské\cite{hunka:bp} i diplomové\cite{hunka:dp} práci, za kterých jsem čerpal následující informace. Tato kapitola popisuje v několika odstavcích její historii.

\subsection{Historický vývoj}
První základy Studentovy berličky byly položeny v zimním semestru 2005/2006, kdy Ing. Jiří Chludil v rámci dobrovolné práce pro předmět Datové struktury a algoritmy zadal vytvoření dynamického systému pro správu cvičení. Prvotní požadavky byly poměrně jednoduché, a tak na sebe první verze, tedy pod názvem Jagu I nenechala slouho čekat. Systém byl po grafické i funkční stránce velmi omezený, ale stal se odrazovým můstkem pro další vývoj.
\begin{figure}[h]
\begin{center}
\includegraphics[width=15cm]{figures/jaguI.png}
\caption{Rozhraní cvičícího - Jagu II}
\label{fig:jaguI}
\end{center}
\end{figure}

Hned v následujícím semestru se přišly nové požadavky na podporu cvičení předmětů Teoretická informatika (TIN) a Aplikace výpočetní techniky (AVT). tyto požadavky zapříčinili nutnost systém v podstatě celý přepsat, bylo nutné přidat například podporu pro zadávání semestrálních prací, zakomponovat bodování docházky a podobně. Nová verze - Jagu 2 - si ovšem stále nesla neduhy svého předchůdce, především co se týče grafického zobrazení a přehlednosti. Na základě zkušeností s vývojem těchto dvou verzí došel Jiří Hunka k závěru, že je nutné vyvinout nový systém, který nebude postaven na bázi pouze jednoho předmětu, ale bude univerzální.

Prvotní verze Studentovy berličky vznikla v rámci bakalářské práce Jiřího Hunky v roce 2007. Stav vycházející z tohoto zpracování je funkční dodnes.

\subsection{Současný stav a budoucnost}

Diplomová práce Jiřího Hunky, ve které se zabýval především uživatelským testováním, končí ne zrovna optimistickými slovy: 
\begin{quotation}
Vizí do budoucna je převést vlastní Studentovu berličku v pouhé GUI a
veškeré další činnosti ponechat na databázi. Toto rozvržení by společně
s využitím protokolu SOAP umožnilo kompletní napojení na jakékoli rozšíření, jinou aplikaci či vlastní spojení i s GUI Studentovy berličky.
Ideálním postupem by bylo celou Studentovu berličku napsat znova. Nejedná se pouze o silná slova, ale o ideální postup, který se navrhuje projektům, které dospějí do této fáze.
\end{quotation}

Současný stav SB je v podstatě typický pro aplikaci napsanou v době, kdy PHP nemělo pořádnou podporu OOP. Aplikace tak nemá žádnou architekturu, prezentační a aplikační logika jsou úzce spojeny a aplikaci nelze snadno rozšiřovat. Z těchto důvodů bylo rozhodnuto, že bude Studentova berlička bude navržena a vytvořena od základů znovu. Na toto téma již bylo zadáno v minulých semestrech několik bakalářských a diplomových prací, které řeší jednotlivé části (jádro, datové úložiště, podporu pluginů, a několik samostatných modulů).

\section{Cíl práce}
Cílem této práce je nalezení, nebo vytvoření datového zdroje, který by poskytl aplikaci Studentova berlička a jejím rozšiřujícím modulům přístup k základním datům o studentech, učitelích a rozvrhu. Toto řešení by mělo být dostatečně univerzální aby pokrylo potřeby nejen Studentovy berličky a jejích modulů, ale i případných dalších aplikací. Základním požadavkem je, kromě univerzálnosti, také možnost vytváření revizí dat a jejich vzájemné porovnávání.

%*****************************************************************************
\chapter{Popis problému, specifikace cíle}

\section{*Současný způsob získávání dat}
Jak popisuje Jiří Hunka ve své bakalářské práci\cite{hunka:bp}, současná verze Studentovy berličky používá pro import textové soubory spravované Ing. Martinem Bílým. Ing. Bílý k nim uvádí tyto informace:

\begin{quotation}
Není mi známo, jak vypadají skutečné databázové tabulky KOSu a jaké všechny atributy obsahují. Vycházím z pravidelně exportovaných XML dat v souboru rz.xml, která předzpracovávám, načítám do jiné databáze, lehce začišťuji a podstatné informace ukládám do následujících souborů v podadresáři data. Každý soubor v datovém adresáři představuje jednu tabulku.

K souborům umožním přístup těm, kteří je potřebují pro svůj projekt, diplomovou či i jinou práci. Uživatel se zavazuje, že data použije právě k těmto účelům.
\end{quotation}

Jednotlivé textové soubory obsahují informace o katedrách, místnostech, studentech, učitelích, rozvrhové lístky a zápisy studentů na předměty. Víceméně jde tedy o kompletní rozvrh. Data jednotlivých sloupců jsou v souborech oddělena čárkou, Ing. Bílý uvádí i jejich význam.

Pro tato data je v databázi Studentovi berličky vytvořena struktura tabulek, která odpovídá struktuře souborů. Konverze dat je prováděna x-krát za semestr ručním spuštěním skriptu. Z tohoto přístupu vyplývá několik nedostatků:


//TODO - zjistit podrobnosti - jak často se importuje


\section{Možnosti získání dat z KOS}

Studijní informační systém KOS je uzavřený systém, ke kterému v podstatě nelze získat přímý aplikační přístup. Samotný KOS totiž neposkytuje žádné dostupné API, pomocí kterého by mohli aplikace jednotlivých fakult a kateder získat alespoň základní data o studentech, učitelích a vyučovaných předmětech. Někteří si proto v minulosti vyžádali nějakou formu generovaného výstupu vhodnou pro svou aplikaci. Jedním takovým exportem je i soubor rz.xml který obsahuje data rozvrhu, tady studenty, učitele, předměty, zápisy do paralelek a další. Každý, kdo chce tento typ dat ve své aplikaci použít, musí tento soubor nějakým způsobem zpracovat - typicky převést do podoby databáze se kterou pak pracuje samotná aplikace. Rozhodl jsem se proto nejdříve zjistit, jak získávání dat řeší některé větší aplikace, a zda-li by se nedali nějakým způsobem využít pro Studentovu berličku.

\section{EDUX}
Aplikace EDUX je webový systém sloužící k organizaci výuky a publikování výukových materiálů na webu pro učitele i studenty. Jde o otevřený systém postavený na aplikaci DokuWiki, jejíchž vlastnosti přebírá a rozšiřuje. V současné době je provozován na serverových clusterech na FEL i na FIT. Studentova berlička by měla v budoucnu s EDUXem úzce spolupracovat, stát se v podstatě jeho plug-inem. EDUX samozřejmě také potřebuje pro své fungování data o rozvrhu z KOS, která získává z exportu rz.xml. Požádal jsem proto Ing. Tomáše Kadlece, který se o EDUX stará, o vyjádření jakým způsobem aplikace zmíněný soubor zpracovává.

EDUX zpracovává rz.xml formou XSL transformace, ze které jsou generovány informace rovnou ve formátu pro Dokuwiki. Tento import je prováděn každou noc, aplikace si zároveň udržuje 90 dní zpět rozdílovou zálohu pro případ problémů, ale jak uvádí Ing. Kadlec, za 1,5 roku provozu EDUXu ji nebylo nutné použít. Studentově berličce tedy nemůže EDUX nabídnout žádný výstup, žádnou \uv{mezi-databázi} vlastně ani nepoužívá.

\section{KOS API}
V letním semestru 2009/2010 vytvořil Jakub Jirůtka v rámci své bakalářské práce\cite{jirutka} aplikaci nazvanou KOSapi. Jde o implementaci webové služby na bázi REST architektury nad zmíněným exportem z KOS rz.xml. Tato aplikace významně zjednodušuje získávání dat rozvrhu z KOS, neboť složité parsování a transformaci XML souboru do přístupnější formy nahrazuje RESTful webová služba s jednoduchou komunikací na bázi požadavek-odpověď nad protokolem HTTP.

Ačkoli tato aplikace řeší několik úskalí práce s rz.xml, nakonec jsem ji nepoužil především z těchto důvodů:
\begin{itemize}
\item Aplikace udržuje vždy jen aktuální verzi dat (z poslední noci). Nelze tedy vytvářet revize a vzájemně je porovnávat, což je jeden z cílů této práce. Bylo by nejspíše nutné implementovat nějakou nadstavbu nad KOSapi, která by toto podporovala.
\item Druhý, a podstatnějším důvod je ten, že v době, kdy jsem na této práci začal pracovat (jaro 2010) nebylo KOSapi ještě hotové a nasazené, k tomu došlo až na podzim. Je ovšem možné, že v budoucnu bude KOSapi nějakým způsobem využito.
\end{itemize}

\section{Vlastní řešení}
Bohužel se mi tedy nepodařilo navíc takovou existující aplikaci, která by splnila všechny požadavky, a proto jsem přistoupil k navržení a implementaci vlastního aplikace. Prvním krokem k tomu bylo přesné specifikování požadavků.
\section{Specifikace požadavků}
\subsection{Funkční požadavky}

\begin{enumerate}
\item Importér z KOS
\begin{enumerate}
\item Importní modul bude převádět data ze souboru rz.xml do podoby relační databáze.
\item Import bude plně automatický.
\item Importní modul bude kontrolovat definované referenční závislosti mezi daty.
\item Importní modul nebude závislý na aktuální struktuře rz.xml, ale bude strukturu dat zpracovávat ze samotného souboru.
\item Zpracování bude logováno aby bylo možné odhalit případná selhání.
\end{enumerate}

\item Webová služba
\begin{enumerate}
\item Systém bude poskytovat data prostřednictvím webové služby.
\item Chyby při zpracování požadavků na WS budou logovány.
\end{enumerate}

\item Poskytovaná data
\begin{enumerate}
\item Systém bude umožňovat definici rozhraní služby pro každou klientskou aplikaci.
\end{enumerate}

\item Aktuálnost dat
\begin{enumerate}
\item Pro každou klientskou aplikaci bude možné definovat přístup ke stále stejné revizi databáze.
\item U revize bude možné na úrovni tabulek definovat stupeň udržované aktuálnosti dat (viz kap. \ref{rozbor})
\item Revize může obsahovat pouze podmnožinu celé databáze.
%\item Revize bude možné porovnávat s aktuální verzí dat.
\item Zároveň systém také umožní přístup ke vždy aktuálním datům (každodenně importovaným).
\end{enumerate}
\end{enumerate}

\subsection{Nefunkční požadavky}
\begin{enumerate}
\item Použitelnost
\begin{enumerate}
\item Celá aplikace bude uživatelsky přívětivá a snadno použitelná.
\item Použitá technologie WS umožní integraci služby s aplikacemi na běžně používaných platformách pro vývoj webových aplikací a plug-inů Studentovy berličky (PHP, C++, Java...).
\end{enumerate}

\item Modulárnost
\begin{enumerate}
\item Celá aplikace se bude skládat z oddělených modulů pro import rz.xml, zpracování požadavků WS a samotné administrace.
\item Moduly na sobě nebudou závislé a nahrazení jednoho modulu by nemělo ovlivnit ostatní.
\end{enumerate}

\item Přístupová práva
\begin{enumerate}
\item Aplikace nebude řešit práva přístupu do administrace služby, to je záležitost webového serveru a nadřazených aplikací.
\item Bude však implementovat nějakou formu identifikace, případně i autorizace klientských aplikací při komunikaci přes WS.
\end{enumerate}

\item Nezávislost na databázi
\begin{enumerate}
\item Systém nebude svázán s konkrétní relační databází.
\end{enumerate}

\end{enumerate}

\subsection{Rozbor požadavků} \label{rozbor}
Některé výše zmíněné požadavky vycházejí ze zkušeností v provozu aktuální verze Studentovy berličky. V této části bych proto rád specifikoval, proč jsou některé požadavky nutné a důležité.

\subsubsection{Verzování dat a porovnávání revizí}
V exportu rz.xml se čas od času vyskytují chyby. To je zkrátka potřeba brát jako fakt a počítat s ním. Jak jsem se dozvěděl od Ing. Chludila a Ing. Kadlece, občas se například stane, že část dat v exportu několik dní chybí a poté se v něm zase objeví. V minulosti také došlo k přečíslování identifikátorů studentů, což způsobilo nefunkčnost Studentovi berličky, která je samozřejmě v databázi využívá. Základním požadavkem na novou aplikaci tedy je, aby dokázala předejít těmto situacím.

To by mělo být zajištěno právě možností vytvářet revize databáze. Každá tabulka v revizi také bude mít možnost definovat stupeň zpracování aktualizace:
\begin{itemize}
\item Neaktualizovat - V tomto případě zůstane tabulka stále stejná z okamžiku vytvoření revize.
\item Aktualizovat pouze data
\item Aktualizovat vše - to znamená strukturu tabulky i data
\end{itemize}
%\end{itemize}
%\begin{itemize}
%\item Neaktualizovat - V tomto případě zůstane tabulka stále stejná z okamžiku vytvoření revize.
%\item Aktualizovat data - Tabulka bude automaticky po importování nové verze rz.xml naplněna aktuálními daty. Tato možnost bude doplněna navíc o údaj maximálního počtu změn v řádcích tabulky, pro které se ještě může provést aktualizace. V případě že bude tento počet překročen, bude upozorněn správce klientské aplikace. Stejně tak v případě že dojde ke změně struktury.
%\item Aktualizovat vše - V takovém případě bude tato tabulka vždy přepsána aktuální verzí z posledního importu. Aplikace v takovém případě dostává stejná data jako by přistupovala přímo k "live" databázi a změny nejsou kontrolovány.
%\end{itemize}

Typickým příkladem proč jsou tyto možnosti potřeba je tento - student si v KOS změní například e-mail a my samozřejmě chceme, aby se změna objevila i v Berličce. Zároveň ale nechceme, aby se u tabulky samovolně mohla změnit struktura a proto nechceme tyto data získávat přímo z \uv{živé} databáze. Správa revizí a jejich aktualizací by také měla pomoci předejít zmíněnému problému s přeindexováním dat.

\subsubsection{Vytváření podmnožin databáze}
Nutnost tohoto požadavku plyne z prostého faktu, že databáze vytvořená ze současného rz.xml obsahuje cca 130 000 záznamů a zabírá cca 30 MB. Typická klientská aplikace ale nepotřebuje všechna data, ale jen jejich část, například informace o jednom předmětu a jeho studentech. Z toho důvodu je zbytečné, vytvářet revizi celá databáze, ale stačí nám pouze její část. U této části musí být také zajištěna konzistence dat , neboť tabulky jsou vzájemně propojeny cizími klíči. Tomuto problému se budu podrobněji věnovat v analýze.

\subsubsection{Vlastní definice rozhraní}
Ze zkušenosti vím, že definice univerzálního rozhraní služby je velmi problematická. Vždy totiž časem přijde takový požadavek, který stávající rozhraní buď vůbec neřeší, nebo jej řeší neefektivně. Takový požadavek pak typicky vyústí v přidání dalších parametrů nebo rovnou nových operací služby, a to \uv{jediné, správné a univerzální} rozhraní se rozrůstá do velkého objemu, a stává se nepřehledným.

Tomu jsem se chtěl vyhnout, proto místo jediného rozhraní aplikace bude umožňovat definici vlastního rozhraní webové služby unikátně pro každou klientskou aplikaci.


%*****************************************************************************
\chapter{*Analýza}

\section{Struktura exportu z KOS}
Jak jsem už zmínil, jediným dostupným způsobem jak získat data rozvrhu je rz.xml. Práce s tímto souborem je celkem obtížná, protože k němu neexistuje naprosto žádná dokumentace. U většiny elementů a atributů lze ale celkem bez problémů rozluštit jejich význam, a tak jsem v podstatě až do doby, než Jakub Jirutky\cite{jirutka} dokončil svou práci neměl potřebu pátrat po významu těch několika zbylých, nerozluštěných. V své práci Jakub Jirůtka zpracovat velmi podrobný popis struktury souboru, a vytvořil tak \textit{de facto} jeho první dokumentaci. 

Soubor obsahuje jeden kořenový element ROZVRH, v něm pak následují elementy jednotlivých \uv{tabulek}, každý s právě jením výskytem. Tabulky pak obsahují data jednotlivých \uv{řádků}, názvy atributů jsou sloupce, hodnoty atributů pak data jednotlivých sloupců. Některé elementy řádků pak ještě obsahují vnořené elementy, kde pak platí že název tagu je názvem sloupce, data v něm pak data příslušného sloupce. Jde například o popis předmětu a požadavků u tabulky PREDMETY, tedy o delší text.

Následující text popisuje stručně strukturu rz.xml, zaměřil jsem se především na rozdíly oproti struktuře kterou popisuje Jakub Jirutka ve své práci\cite{jirutka}. Jednotlivé elementy nejsou v následujícím textu uvedeny v pořadí v jakém se nacházejí v exportu, ale podle logických souvislostí.

\subsection{ROZVRH}
Element ROZVRH je kořenovým elementem XML stromu dokumentu. Obsahuje základní informace o semestrech:
\begin{itemize}
\item \texttt{semestr, aktsem, aktsem2} - Kódy semestrů pro které se jsou data v rz.xml generována. V průběhu semestru je ve všech atributech uvedený stejný kód (např. B101), v době kdy se tvoří rozvrh se kódy různí.
\item \texttt{nazev\_semestru} - Textový název aktuálního semestru (odpovídá semestru v atributu \texttt{semestr})
\item \texttt{zacatek\_sem, konec\_sem} - Obsahuje datum začátku, resp. konce aktuálního semestru dle harmonogramu školního roku. Formát data je \texttt{dd.mm.yyyy}.
\end{itemize}

\subsection{KATEDRY}
V databázi KOS tvoří jsou rektorát, fakulty a katedry umístěny v jedné tabulce a tvoří stromovou strukturu středisek. Záznamy jsou spojeny relací \texttt{nadrizena=id}, přičemž fakultu od katedry lze snadno rozlišit tím, že fakulta má nadrizena=1 (id rektorátu). V souboru rz.xml nalezneme ale jen FEL a jeho katedry, takže tuto relaci nemůžeme explicitně přenést do databáze.

Element obsahuje atributy \texttt{id, nadrizena, nazev\_cz, nazev\_en} (český a anglický název) a číselný \texttt{kod} střediska.

\subsection{MISTNOSTI}
Element obsahuje místnosti ve kterých probíhá výuka. Atributy jsou \texttt{id, cislo} (např. KN:E-107), lokalitu (\texttt{lok}) a odkaz na katedru, která má místnost ve správě (\texttt{kat\_id}).

\subsection{PREDMETY}
Element PREDMETY obsahuje všechny vypsané předměty v semestrech uvedených v elementu ROZVRH. Protože jsou samozřejmě stejné předměty vypisovány v různých semestrech, jsou jednotlivé záznamy identifikované složeným primárním klíčem \texttt{id, sem\_id}.

Element obsahuje následující atributy a elementy:

\begin{itemize}
\item \texttt{id, sem\_id} identifikátor záznamu
\item \texttt{kod} Kód předmětu, např. Y36BAP.
\item \texttt{garanti} Identifikátory učitelů, kteří jsou garanty předmětu. Obsahuje hodnoty oddělené čárkou.
\item \texttt{kat\_id, pr\_kat\_id} Identifikátor katedry, která předmět zajišťuje (oba atributy obsahují stejnou hodnotu).
\item \texttt{nazev, nazev\_an} Český, respektive anglický název předmětu (ano, anglický název má skutečně suffix \textit{an}, nikoliv \textit{en})
\item \texttt{rozsah} Hodinová dotace předmětu.
\item \texttt{pr\_forma\_studia} Forma studia pro kterou je předmět určen.
\item \texttt{pr\_kredity} Počet kreditů za absolvování předmětu.
\item \texttt{pr\_zpuszak} Způsob zakončení předmětu, hodnoty [KZ|Z|ZK|Z,ZK|NIC].
\item \texttt{pr.kapacita, pr\_obsazeno, pr\_zapnuto} Kapacita předmětu, počet zapsaných a příznak určující, zda-li je zapnuto hlídání překročení kapacity. (ano, atribut kapacita skutečně obsahuje v názvu tečku, nikoliv podtržítko).
\item \texttt{pr\_program\_id} ID programů ve kterých je možné předmět zapsat. Obsahuje hodnoty oddělené čárkou.
\item \texttt{pr\_p\_blok, pr\_c\_blok, pr\_l\_blok} Toto jsou pomocné atributy, které určují, kolik rozvrhových lístků má být vygenerováno pro přednášky, cvičení a laboratoře.
\item \texttt{pozadavky, poznamka} Toto jsou vnořené elementy v elementu řádku, obsahují textový popis požadavků na předmět, respektive textovou poznámku.
\end{itemize}
Na rozdíl od toho, co popisuje Jakub Jirůtka, v současném rz.xml se nevyskytuje atribut \texttt{pr\_znamky\_hlas}.

\subsection{Učitelé a studenti}
V KOSu existují samostatné tabulky osob, učitelů a studentů. V exportu jsou ovšem data osoby a učitele, resp. studenta sloučeny do jednoho elementu, což může přinést určité problémy.

\subsubsection{UCITELE}
U učitelů existuje jen drobná záludnost, a to že někteří externí učitelé, kteří nemají přístup do KOS nemají vyplněný atribut \texttt{login}. Element obsahuje tyto atributy:
\begin{itemize}
\item \texttt{id}
\item \texttt{kat\_id} Identifikátor katedry, ke které učitel patří. Obsahuje i katedry, které v elementu \texttt{KATEDRY} nejsou, a také mnoho \texttt{NULL} hodnot.
\item \texttt{login}
\item \texttt{jmeno, prijmenu, titul\_pred, titul\_za}
\item \texttt{uc\_email} Obsahuje nevalidní e-maily a množství nevyplněných hodnot.
\item \texttt{osobni\_cislo}
\item \texttt{osoba\_eid}
\end{itemize}

\subsubsection{STUDENTI}
Záznamy v této části jsou identifikovány složeným klíčem \texttt{id, stud\_id} (id studia). Jedna osoba totiž může být zapsána do několika studijních oborů najednou.
Element obsahuje tyto atributy:

\begin{itemize}
\item \texttt{id, stud\_id} identifikátor záznamu
\item \texttt{jmeno, prijmeni, titul, titul\_za} Celé jméno včetně titulů (všimněte si, že na rozdíl od záznamů učitelů nemá titul před jménem suffix).
\item \texttt{obor} Textový název studijního oboru, např. \uv{Web a multimedia (bakalářský)}
\item \texttt{osoba\_eid}
\item \texttt{osoba\_id} Obsahuje stejné hodnoty jako atribut \texttt{id}
\item \texttt{rocnik}
\item \texttt{skupina} Studijní skupina
\item \texttt{stud\_email}
\item \texttt{stlan\_id} ID studijního plánu
\end{itemize}

\subsection{ZAPISY\_PREDMETU}
Obsahuje data spojové tabulky pro zápisy studentů do předmětů. Obsahuje celý identifikátor předmětu \texttt{predmet\_id, sem\_id} a identifikátor studenta \texttt{student\_id}.

\subsection{LISTKY}
Element LISTKY obsahuje záznamy o vypsaných paralelkách přednášek a cvičení přiřazených do konkrétních dní, časů a místností. Existují zde rozvrhové lístky předmětů a pak speciální události jiného typu - například porady, semináře, výuka na jiných školách a podobně. Tyto speciální záznamy lze rozlišit tím, že atribut \texttt{typ\_vyuky} obsahuje hodnotu \uv{J}. V tomto elementu najdeme tyto údaje:
\begin{itemize}
\item \texttt{id}
\item \texttt{den\_cis} Den v týdnu, rozsah 1-7.
\item \texttt{hodina} Číslo vyučovací hodiny, kdy lístek začíná.
\item \texttt{pocet\_hodin} Počet vyučovacích hodin lístku.
\item \texttt{katedra1\_id} Katedra, která vlastní lístek.
\item \texttt{katedra2\_id} Katedra, která zajišťuje výuku.
\item \texttt{mistnost\_id} Místnost, kde se výuka probíhá.
\item \texttt{typ\_vyuky} Identifikace, zda jde o přednášku (P), cvičení (C), laboratoř (L), nebo speciální jinou událost (J).
\item \texttt{pno, cno, lno} Jde o identifikátory ke které paralelce lístek přísluší. Např. \texttt{pno=1, cno=202, lno=NULL} znamená, že paralelku cvičení 202 lze zapsat jen s přednáškovou paralelkou 1.
\item \texttt{predmet\_id} ID předmětu, ke kterému lístek přísluší. Element lístky neobsahuje druhou část složeného klíče předmětu \texttt{sem\_id}, protože se exportují jen lístky k semestru uvedeném v atributu \texttt{semestr} elementu \texttt{ROZVRH}.
\item \texttt{ucitel1\_id, ucitel2\_id} ID učitelů, kteří zajišťují výuku. Pokud jde o lístek typu jiná událost, je vyplněn pouze \texttt{ucitel1\_id} a jde v tomto případě o vlastníka události.
\item \texttt{sudy\_lichy} Příznak, zda-li jde o lístek, který je platný lichá nebo sudé týdny (pokud vždy, obsahuje atribut \texttt{NULL}).
\item \texttt{text} U speciálních jiných událostí obsahuje její popis.
\item \texttt{paralelka\_kod}
\item \texttt{par\_jmeno} Jméno paralelky, většinou obsahuje nějakou poznámku.

\end{itemize}

\subsection{LISTKY\_STUDENTU}
Obsahuje data spojové tabulky pro zápisy studentů do rozvrhu. Obsahuje identifikátor lístku \texttt{listek\_id} a identifikátor studenta \texttt{student\_id}. Mimochodem jde o datově největší element v exportu.

\subsection{JEDNORAZOVE\_TERMINY}
Tento element obsahuje informace o jednorázových akcích, která se například používají pro hromadné kontroly semestrálních prací nebo udělování zápočtů.

Najdeme zde tyto atributy:
\begin{itemize}
\item \texttt{id}
\item \texttt{nazev} Název akce
\item \texttt{datum, cas} Datum a čas konání akce.
\item \texttt{fakulta\_id}
\item \texttt{katedra\_id}
\item \texttt{misto\_id}
\item \texttt{predmet\_id, sem\_id} Identifikátor předmětu, kterého se akce týká.
\item \texttt{kapacita, obsazeno} Kapacita akce a počet přihlášených.
\item \texttt{uzaverka} datum uzavření přihlašování (dd.mm.yyyy).
\item \texttt{vypsal\_id}
\item \texttt{poznamka}
\end{itemize}

\subsection{ZAPISY\_JA}
Obsahuje data spojové tabulky pro zápisy studentů na jednorázovou akci. Obsahuje identifikátor akce \texttt{termin\_id} a identifikátor studenta \texttt{student\_id} a \texttt{id}.

\subsection{VYPSANE\_TERMINY}
Element obsahuje vypsané termíny zkoušek k předmětům. Najdeme v něm následující atributy:

\begin{itemize}
\item \texttt{id}
\item \texttt{vypsal\_id} 
\item \texttt{zacatek} Datum a čas začátku termínu (formát dd.mm.yyyy HH:MM)
\item \texttt{konec} Datum a čas konce termínu (formát dd.mm.yyyy HH:MM)
\item \texttt{predmet\_id, sem\_id} Identifikátor předmětu
\item \texttt{misto\_id} ID místnosti, kde by se termín měla konat. Často není vůbec vyplněné a skutečná místnost je uvedená v poznámce.
\item \texttt{poznamka}
\end{itemize}

\subsection{ZKOUSEJICI}
Obsahuje data spojové tabulky určující, kteří učitelé předmět zkouší. Obsahuje identifikátor předmětu \texttt{predmet\_id, sem\_id} a identifikátor učitele \texttt{osoba\_id}.

\subsection{CASOVA\_OKENKA}
Tento element obsahuje data okének, které se používají pro plánování rozvrhu. Vždy když je vytvořena nějaká akce (např. lístek) ozančí se místnost jako zabraná právě vytvořením časového okénka. Žádný jiný element v exportu ale neobsahuje odkaz na záznamy v tomto elementu a ani tento element neobsahuje identifikaci místnosti ke které se vztahuje. Jeho existence tedy naprosto postrádá smysl.

Element obsahuje atributy \texttt{id}, datum ve formátu dd.mm.yyyy (\texttt{den}), času začátku a konce okénka (\texttt{zacatek}, \texttt{konec}) a poznámku (\texttt{poznamka}).

\section{Verzování databáze}
\subsection{Zachování integrity dat}




%*****************************************************************************
\chapter{Návrh řešení}


\section{Použité technologie}
Pro implementaci aplikace byl použit skriptovací jazyk PHP a databázový stroj MySQL, a to především z toho důvodu, že s nimi mám nejvíce praktických zkušeností. Nezanedbatelným argumentem je také to, že stávající Studentova berlička je také využívá.

\subsection{PHP}
PHP je velmi oblíbený skriptovací programovací jazyk používaný především pro tvorbu dynamických webových stránek. Od verze 5 disponuje také dobrou podporou objektově orientovaného programování. Přestože si jazyk s sebou z minulosti nese spoustu nevýhod, jde dnes nejspíše o nejpoužívanější jazyk pro tvorbu webů. Poslední významná verze PHP 5.3 rozšířila možnosti jazyka také o deklaraci jmenných prostorů, anonymní funkce, řetězení výjimek a mnoho dalších menších změn.

\subsection{MySQL}
MySQL je multiplatformní relační databázový systém od švédské společnosti MySQL AB. Pro svou jednoduchou instalaci, konfiguraci, výkon a také proto, že je volně dostupné pod GNU GPL, je na webových serverech velmi oblíbená, především v kombinaci s Apache a PHP. MySQL bylo od počátku svého vývoje optimalizováno především pro výkon, a to i za cenu zjednodušení funkčnosti, například podpora definice cizích klíčů a transakcí byla přidána až ve verzi 3.23, šest let od prvního vydání. MySQL podporuje většinu ze standartu SQL:99, doplněnou, stejně jako u většiny ostatních DBMS, o vlastní rozšíření.

\subsection{Nette Framework}
Nette je český framework pro tvorbu webových aplikací v PHP 5. Zaměřuje se především na eliminaci bezpečnostních rizik a vede programátory k čistému návrhu aplikace s důrazem na budoucí rozšiřitelnost. Využívá událostmi řízeného programování a principy třívrstvé architektury Model-View-Presenter, DRY\footnote{neopakovat se} a KISS\footnote{řešit vše jednoduše}. Mezi jeho silné stránky patří také výkonný nástroj pro odchytávání chyb (Nette\verb|\|Debug), který místo prosté textové informace PHP zobrazí celou stránku s detailním výpisem místa, kde k chybě došlo, call stacku a důležitých proměnných aplikace. Dalšími velmi užitečnými částmi frameworku jsou formuláře se silným validačním jazykem, routování, které slouží jak pro zpracování požadavků, tak i pro generování odkazů, a v neposlední řadě šablonovací jazyk Latte.

Za dva roky veřejné existence si Nette získalo velkou popularitu na česko-slovenském webu, včetně komerčních aplikací. Původním autorem Nette je David Grudl, dnes se o jeho vývoj stará komunita vývojářů Nette Foundation. Framework je volně dostupný pod licencí GNU/GPL a licencí Nette, která je obdobou BSD licence\cite{nette:licence}.

\subsection{Databázová abstrakce}
Nutnost použití databázové abstrakce plyne z NP4a. Přímo v distribučním balíku PHP najdeme rozšíření PDO, které poskytuje abstrakční vrstvu a odstínění od konkrétní databáze. Já jsem se rozhodl nepoužít přímo PDO, ale knihovnu dibi.
\subsubsection{dibi}
Dibi je, jak se dočteme v podtitulu názvu \uv{tiny 'n' smart database layer} (malá a chytrá databázová vrstva). Jde o knihovnu, jejímž hlavním cílem je zjednodušit zápis SQL příkazů a ulehčit rutinní úlohy programátora, například získání výsledků jako dvourozměrné pole, vytvoření pole výsledků indexovaného podle klíče z databáze a podobně. Dibi také pomáhá předcházet zranitelnosti webů SQL Injection, neboť všechny parametry, které do dotazů vstupují jsou automaticky escapovány syntaxí podle použité databáze. Lze ho použít jednak jako nadstavbu PDO, tak přímo s nativními PHP rozšířeními pro jednotlivé databázové stroje.
\label{idatabasemanager}
\subsubsection{Rozhraní pro správu databází}
PDO i dibi poskytují abstrakci, která je dostačující pro běžnou aplikaci, které pracuje s jednou databází\footnote{Databází zde myslím kolekci tabulek. U MySQL se používá termín databáze, například u PostgreSQL by byl správný termín \uv{schéma}}. S jinými databázemi lze samozřejmě pracovat prostřednictvím \uv{tečkové} notace {\tt databáze.tabulka.sloupec}, ovšem v momentě, kdy je potřeba nastavit některou databázi jako výchozí (např. u MySQL příkazem {\tt USE database}), nenabízí PDO ani dibi žádnou možnost. Totéž pak platí i pro možnosti pro vytváření databáze a tabulek, jejich mazání, úpravy struktury a podobně.  A to je přesně to, co v této aplikaci potřebuji, neboť každá revize odpovídá samostatné databázi. Přímé zapsání těchto příkazů by samozřejmě způsobilo svázání aplikace s konkrétním DBMS, neboť syntaxe těchto příkazů ke u jednotlivých strojů různá.

Z toho důvodu jsem v aplikaci vytvořil jednoduché rozhraní {\tt IDatabaseManager} které odstiňuje ostatní části od specifik databázového stroje. Převod systému pod jiný databázový stroj tedy spočívá jen v implementaci tohoto rozhraní pro konkrétní DBMS a změnu jednoho řádku v konfiguračním souboru.

\subsection{Webové služby}
Webové služby jsou technologický systém, který umožňuje propojit libovolné aplikace na různých platformách pomocí široce podporovaného a vždy dostupného protokolu HTTP. V dnešní době jsou běžně používány dva způsoby implementace webových služeb - REST a SOAP.

\subsubsection{REST}
Architektura REST (Representational State Transfer) zažila v posledních letech obrovskou vlnu popularizace. Není to vlastně ani standart, ale jen architektura rozhraní. Na rozdíl od procedurálního SOAP je REST orientován datově. Každý datový zdroj má specifické URI a pro komunikaci se službou se využívají metody protokolu HTTP GET, POST, PUT a DELETE. Díky tomu jsou REST služby snadno škálovatelné a jednoduché na pochopení. Velkou nevýhodou ovšem je chybějící podpora na úrovni programovacích jazyků (narozdíl od SOAP).


\subsubsection{SOAP}
SOAP (Simple Object Access Protocol) vychází ze specifikace XML-RPC a přebírá jeho hlavní rys - formát přenášených zpráv a jeho platformní nezávislost. SOAP umožňuje zasílání XML zpráv mezi dvěma aplikacemi a pracuje tedy na principu peer-to-peer. Mnohem častěji se ale technologie používá jako vzdálené volání procedur (RPC), tedy ve schématu klient-server. Stejně jako REST probíhá komunikace nejčastěji protokolem HTTP, ačkoli specifikace protokol nijak nepředepisuje a lze použít i SMTP, FTP a jiné.

Dalším hlavním rysem služby na principu SOAP je možnost definice jejího rozhraní pomocí WSDL schématu. To popisuje procedury, která služba nabízí, typy jejich parametrů a návratových hodnot. I když je zpřístupnění WSDL popisu velmi užitečné a praktické, není vyžadováno a služba může být využívána i bez něj.

Protože narozdíl od REST existuje v distribuci PHP knihovna pro práci se SOAP, a protože některé již existující části Studentovy berličky SOAP používají, zvolil jsem pro mají webovou službu také tento protokol.

\section{Architektura aplikace}
Jak jsem již uvedl v části o použitých technologiích, aplikace je postavena na Nette Frameworku, který využívá třívrstvou architekturu Model-View-Presenter. V této části najdete, co je to vlastně návrhový vzor MVP a jeho rozdíly oproti známějšímu vzoru Model-View-Controller.

\subsection{Model-View-Controller}
Model-View-Presenter (dále jen MVP) je odvozený od mnohem známějšího návrhového vzoru Model-View-Controller (dále jen MVC). MVC definuje rozdělení aplikace do tří samostatných vrstev tak, že modifikace v jedné vrstvě má minimální vliv na vrstvy ostatní. Aplikace se rozdělí na:
\begin{itemize}
\item Model - je doménově specifická reprezentace informací, s nimiž aplikace pracuje. V běžné webové aplikaci se modelová část stará o komunikaci s databází a persistenci dat.
\item View (pohled) - jde o část aplikace, se kterou komunikuje uživatel, u webové aplikace jde tedy o šablony a výsledný HTML kód který se zobrazuje v prohlížeči.
\item Controller (řadič, ovladač) - reaguje na události (většinou pocházející od uživatele), a zajišťuje změny v pohledu a v modelu
\end{itemize}
\begin{figure}[h]
\begin{center}
\includegraphics[width=10cm]{figures/mvc.png}
\caption{Architektura Model-View-Controller}
\label{fig:mvc}
\end{center}
\end{figure}
Tento princip poprvé popsal Trygve Reenskaug v roce 1979\cite{mvc-original}. Základní myšlenkou MVC je tedy rozdělení aplikace do tří modulů. Model pouze poskytuje své rozhraní a neví nic o Controlleru ani o View. View o modelu také vědět nemusí (pasivní view), nebo naopak může získávat data přímo z něj, dle zvolené koncepce. Controller potom slouží k propojení zbývajících dvou částí a zpracování požadavků od uživatele.

MVC ovšem nelze chápat dogmaticky. Už v době svého vzniku bylo jeho pojetí velmi různorodé, první návrh vlastně ani není MVC, ale MVCE, kde \uv{E} znamená Editor. K čemu slouží? Jde v podstatě o totéž, co bylo v první implementaci MVC označeno jako Controller a je dokonce společně s View reprezentováno jednou třídou. Ve své prapůvodní podobně MVC vlastně nepoužívá nikdo, role a vztahy jednotlivých vrstev se chápou velmi volně. To je také důvod, proč se Nette Framework hlásí k MVC jen jako k duševně spřízněné architektuře.

\subsection{Model-View-Presenter}

Mnohem výstižnější pro logiku Nette Frameworku je méně známý vzor Model-View-Presenter. V Nette totiž platí myšlenka \uv{nechť je odkaz totéž, co zavolání funkce}\cite{zdrojak:nette}. View už není prostý HTML kód jako na počátku, ale jeho součástí je také Javascript pro obsluhu uživatelských požadavků a AJAXu. Presenter je tedy volán z View, nikoli přímo uživatelem. Došlo tak oproti MVC k posunutí významu vrstev:
\begin{figure}[h]
\begin{center}
\includegraphics[width=10cm]{figures/mvp.png}
\caption{Architektura Model-View-Presenter}
\label{fig:mvp}
\end{center}
\end{figure}
\begin{itemize}
\item Model se stará stále o to samé, tedy o persistenci dat.
\item View navíc oproti MVC zpracovává uživatelský vstup. Typicky například po kliknutí na odkaz volá nějakou funkci Presnteru. Aplikační logika ve View je většinou chybou.
\item Presenter obsluhuje požadavky z View a manipuluje s Modelem. Spará se vlastně o \uv{prezentování} výsledků (odtud název Presenter).
\end{itemize}

Reálně je problém rozlišit, jestli má aplikace architekturu MVC nebo MVP, protože fakticky se vzory liší jen v detailech. Nette používá označení Presenter, naproti tomu například v Zend Frameworku najdeme Controllery, které se starají o totéž. Další informace o MVP lze najít například v \cite{mvp}.


\section{*Načtení XML exportu}
\subsection{*Cizí klíče a referenční závislosti}




%*****************************************************************************
\chapter{Implementace}
Tato kapitola obsahuje postup nasazení aplikace na produkční server, popis jejích možností konfigurace a také příklad na klienta webové služby.
 
\section{Nasazení aplikace}
\subsection{Požadavky}
\begin{itemize}
\item Webový server s podporou PHP (doporučeným serverem, na kterém byla aplikace testována je Apache HTTPD 2.2 s PHP jako SAPI modulem).
\item PHP verze 5.3 nebo vyšší (aplikace využívá jmenné prostory a anonymní funkce). Jsou vyžadovány moduly PHP \textit{soap, curl, mbstring}.
\item Relační databázový stroj (doporučeno MySQL nebo PostgreSQL), a odpovídající modul PHP pro jeho podporu (možné použít PDO i nativní rozšíření).
\item Libovolný operační systém (testováno na platformě Windows a Debian GNU/Linux).
\end{itemize}
V případě použití jiného databázového stroje než MySQL je nutné doplnit do aplikace podporu implementaci rozhraní pro správu databází pod tímto RDBMS. Více viz kap. \ref{idatabasemanager}.

\subsection{Konfigurace}
Konfigurace aplikace je umístěna na místě standardním pro Nette aplikace - v souboru \textit{/app/config.ini}. Je velmi důležité, aby tento soubor zůstal nepřístupný pro přístup z webu, protože obsahuje citlivé informace. Z webu by měl být přístupný pouze adresář \textit{/www/}.

Při nasazování aplikace je nutné soubor config.ini vytvořit, distribuce obsahuje ve stejné složce připravený soubor config-default.ini, který obsahuje výchozí nastavení.

Konfigurační soubor Nette aplikace je rozdělen na části \textit{common, development} a \textit{production}. Jednotlivé sekce je možné v jiných \uv{dědit}, podobně jako třídy programu. K tomu se používá notace se špičatou závorkou (např. \texttt{[production < common]}). Uvedený příklad znamená, že sekce \textit{production} rozšiřuje sekci \textit{common}. Stejně jako u dědění tříd je možné přepsat nastavení nadřazené sekce, čehož se využívá například pro rozdílná nastavení připojení k databázi na vývojovém a produkčním serveru.

Konfigurační soubor obsahuje nastavení běhového prostředí aplikace a jejích služeb. Zde neuvádím všechny direktivy, které se v souboru vyskytují, ale jen ty které mohou být užitečné právě při nasazování aplikace. Zde neuvedené direktivy doporučuji neměnit, můžete tím znefunkčnit celou aplikaci!

Nastavení databáze:
\begin{itemize}
\item \textbf{database.driver} Databázový driver pro připojení. Lze použít jak nativní driver \textit{mysql, mysqli, pgsql} nebo \textit{pdo}.
\item \textbf{database.dsn} V případě použití PDO obsahuje DSN řetězec pro připojení k databázi. Při použití nativních ovladačů je ignorován.
\item \textbf{database.host} Adresa databázového stroje.
\item \textbf{database.username} Uživatelské jméno pro připojení
\item \textbf{database.password} Heslo pro připojení
\item \textbf{database.charset} Kódování znaků
\item \textbf{database.lazy} V případě nastavení této direktivy na TRUE je připojení k databázi vytvořeno až při provedení prvního dotazu.
\item \textbf{service.IDatabaseManager} Nastavení jména třídy, implementující rozhraní \texttt{IDatabaseManager}, které se má použít pro práci s databází. Výchozí hodnotou je implementace pro \textit{MySQLDatabaseManager} pro databázový stroj MySQL.
\end{itemize}
Nastavení celé sekce \textit{database} vychází z API třídy \textit{DibiConnection}\cite{dibiconnection}, lze použít veškeré jeho možnosti. Na tomto místě bych také zdůraznil, že databázový uživatel musí mít práva pro vytváření, změnu a odstraňování databází a jejích tabulek.

Nastavení importního modulu:
\begin{itemize}
\item \textbf{xml.remoteURL} URL adresa vzdáleného zdroje, odkud se má rz.xml stáhnout.
\item \textbf{xml.login} Uživatelské jméno použité pro HTTP autentizaci.
\item \textbf{xml.password} Heslo použité pro HTTP autentizaci.
\item \textbf{xml.localRepository} Cesta k lokálnímu úložišti XML souborů.
\item \textbf{xml.liveDatabase} Název databáze, do které bude aktuální verze rz.xml převedena.
\end{itemize}


\section{Ukázková implementace klienta služby}
Protože webová služba umožňuje definici vlastního rozhraní služby, budu pro následující příklady předpokládat, že v aplikaci zadáno následující nastavení:
\begin{itemize}
\item Klient je identifikován jménem \uv{berlicka} a heslem \uv{test}.
\item Klient může načíst data z revize databáze pod názvem \uv{testing}.
\item Klient má definovánu jednu proceduru služby s názvem \textit{getStudentsInfo}, která má jako parametr pole studentských uživatelských jmen a vrací asociativní pole informací o studentech (z tabulky \textit{studenti}) indexované osobním ID.
\end{itemize}

Identifikace klientských aplikací probíhá zavoláním procedury \texttt{authenticate(jmeno, heslo)}. Je nutné volat tuto proceduru jako první, jinak služba vrací výjimku \textit{SoapFault}.

Další pevnou součástí rozhraní služby je \texttt{useRevision}, která slouží k výběru aktivní revize. Revizi lze bez problémů změnit během komunikace se službou, čili různá volání procedur mohou pracovat s různými revizemi. Pokud není před zavoláním nějaké procedury vybrána revize voláním \texttt{useRevision}, automaticky se použije revize nastavená pro klienta jako výchozí.

V distribuci PHP najdeme již zabudovanou podporu SOAP (přesněji je v rozšíření \textit{soap}, které je součástí distribuce), takže klienta vytvoříme snadno pomocí třídy SoapClient. Protože služba neposkytuje WSDL, je první parametr NULL a teprve druhým parametrem je určena adresa služby, tak jak to radí manuál PHP\cite{php:soap}. Nyní již slíbený příklad:
\begin{verbatim}
<?php

$soap = new SoapClient(NULL, array(
    "location" => 'http://URL_SERVERU/soap/',
    "uri" => 'http://URL_SERVERU/soap/'));
try {
    $soap->authenticate('berlicka', 'test');
    $soap->useRevision('testing');

    $return = $soap->getStudentsInfo(array('langeja1'));
    var_dump($return);
} catch (SoapFault $e) {
    echo $soap->getLastError();
}

?>
\end{verbatim}

Pokud vše proběhne v pořádku, obdržíme odpověď:

\begin{verbatim}
array(1) {
   355981000 => array(14) {
      "id" => "355981000" (9)
      "jmeno" => "Jan" (3)
      "login" => "langeja1" (8)
      "obor" => "Web a multimedia (bakalářský)" (32)
      "osoba_id" => "355981000" (9)
      "prijmeni" => "Langer" (6)
      "rocnik" => "3"
      "skupina" => "72" (2)
      "stlan_id" => "10201404" (8)
      "stud_email" => "langeja1@fel.cvut.cz" (20)
      "stud_id" => "15736704" (8)
      "titul" => NULL
      "titul_za" => NULL
   }
}
\end{verbatim}

\subsection{Zpracování chyb}
Jakákoliv chyba při zpracování požadavku (například v SQL dotazu, nesprávný formát parametrů nebo chybná identifikace)na straně serveru služby je indikována vyhozením výjimky. Na klientské straně se to projeví výjimkou \textit{SoapFault}, je tedy velmi vhodné ji zachytávat blokem try-catch. Ani po vyhození této speciální výjimky, ale nedojde k uzavření spojení a tak je možné zjistit textový popis chyby voláním procedury \texttt{getLastError()}. Stejný text je na straně serveru také zalogován a je možná ho najít v administračním rozhraní.


%*****************************************************************************
\chapter{*Testování}
\begin{itemize}
 \item Způsob, průběh a výsledky testování.
 \item Srovnání s existujícími řešeními, pokud jsou známy.
\end{itemize}

%*****************************************************************************
\chapter{*Závěr}

\begin{itemize}
\item Zhodnocení splnění cílů DP/BP a  vlastního přínosu práce (při formulaci je třeba vzít v potaz zadání práce).
\item Diskuse dalšího možného pokračování práce.
\end{itemize} 

%*****************************************************************************
% Seznam literatury je v samostatnem souboru reference.bib. Ten
% upravte dle vlastnich potreb, potom zpracujte (a do textu
% zapracujte) pomoci prikazu bibtex a nasledne pdflatex (nebo
% latex). Druhy z nich alespon 2x, aby se poresily odkazy.

% originally following specification for bibliography formating was used
%\bibliographystyle{abbrv}

% Here is an improvment by Petr Dlouhy (April 2010).
% It is mainly for supervisors who expect Czech fomrating rules for references
% Additional feature is live url addresses to sources from your pdf file
% It requires the file csplainnat.bst (included in this sample zipfile).

\bibliographystyle{csplainnat}

%bibliographystyle{plain}
%\bibliographystyle{psc}
{
%JZ: 11.12.2008 Kdo chce mit v techto ukazkovych odkazech take odkaz na CSTeX:
%\def\CS{$\cal C\kern-0.1667em\lower.5ex\hbox{$\cal S$}\kern-0.075em $}
\bibliography{reference}
}

% M. Dušek radi:
%\bibliographystyle{alpha}
% kdy citace ma tvar [AutorRok] (napriklad [Cook97]). Sice to asi neni  podle ceske normy (BTW BibTeX stejne neodpovida ceske norme), ale je to nejprehlednejsi.
% 3.5.2009 JZ polemizuje: BibTeX neobvinujte, napiste a poskytnete nam styl (.bst) splnujici citacni normu CSN/ISO.

%*****************************************************************************
%*****************************************************************************
\appendix

\chapter{Testování zaplnění stránky a odsazení odstavců}
\textbf{\large Tato příloha nebude součástí vaší práce. 
Slouží pouze jako příklad formátování textu.}

\section*{}
Určitě existuje nějaká pěkná latinská věta, která se k tomuhle testování používá, ale co mají dělat ti, kteří se nikdy latinsky neučili? Určitě existuje nějaká pěkná latinská věta, která se k tomuhle testování používá, ale co mají dělat ti, kteří se nikdy latinsky neučili? Určitě existuje nějaká pěkná latinská věta, která se k tomuhle testování používá, ale co mají dělat ti, kteří se nikdy latinsky neučili?

Určitě existuje nějaká pěkná latinská věta, která se k tomuhle testování používá, ale co mají dělat ti, kteří se nikdy latinsky neučili? Určitě existuje nějaká pěkná latinská věta, která se k tomuhle testování používá, ale co mají dělat ti, kteří se nikdy latinsky neučili? Určitě existuje nějaká pěkná latinská věta, která se k tomuhle testování používá, ale co mají dělat ti, kteří se nikdy latinsky neučili?

Určitě existuje nějaká pěkná latinská věta, která se k tomuhle testování používá, ale co mají dělat ti, kteří se nikdy latinsky neučili? Určitě existuje nějaká pěkná latinská věta, která se k tomuhle testování používá, ale co mají dělat ti, kteří se nikdy latinsky neučili? Určitě existuje nějaká pěkná latinská věta, která se k tomuhle testování používá, ale co mají dělat ti, kteří se nikdy latinsky neučili?

Určitě existuje nějaká pěkná latinská věta, která se k tomuhle testování používá, ale co mají dělat ti, kteří se nikdy latinsky neučili? Určitě existuje nějaká pěkná latinská věta, která se k tomuhle testování používá, ale co mají dělat ti, kteří se nikdy latinsky neučili? Určitě existuje nějaká pěkná latinská věta, která se k tomuhle testování používá, ale co mají dělat ti, kteří se nikdy latinsky neučili? Určitě existuje nějaká pěkná latinská věta, která se k tomuhle testování používá, ale co mají dělat ti, kteří se nikdy latinsky neučili? Určitě existuje nějaká pěkná latinská věta, která se k tomuhle testování používá, ale co mají dělat ti, kteří se nikdy latinsky neučili? Určitě existuje nějaká pěkná latinská věta, která se k tomuhle testování používá, ale co mají dělat ti, kteří se nikdy latinsky neučili?

Určitě existuje nějaká pěkná latinská věta, která se k tomuhle testování používá, ale co mají dělat ti, kteří se nikdy latinsky neučili? Určitě existuje nějaká pěkná latinská věta, která se k tomuhle testování používá, ale co mají dělat ti, kteří se nikdy latinsky neučili?

Určitě existuje nějaká pěkná latinská věta, která se k tomuhle testování používá, ale co mají dělat ti, kteří se nikdy latinsky neučili? Určitě existuje nějaká pěkná latinská věta, která se k tomuhle testování používá, ale co mají dělat ti, kteří se nikdy latinsky neučili? Určitě existuje nějaká pěkná latinská věta, která se k tomuhle testování používá, ale co mají dělat ti, kteří se nikdy latinsky neučili? Určitě existuje nějaká pěkná latinská věta, která se k tomuhle testování používá, ale co mají dělat ti, kteří se nikdy latinsky neučili? Určitě existuje nějaká pěkná latinská věta, která se k tomuhle testování používá, ale co mají dělat ti, kteří se nikdy latinsky neučili?

Určitě existuje nějaká pěkná latinská věta, která se k tomuhle testování používá, ale co mají dělat ti, kteří se nikdy latinsky neučili? Určitě existuje nějaká pěkná latinská věta, která se k tomuhle testování používá, ale co mají dělat ti, kteří se nikdy latinsky neučili? Určitě existuje nějaká pěkná latinská věta, která se k tomuhle testování používá, ale co mají dělat ti, kteří se nikdy latinsky neučili? Určitě existuje nějaká pěkná latinská věta, která se k tomuhle testování používá, ale co mají dělat ti, kteří se nikdy latinsky neučili? Určitě existuje nějaká pěkná latinská věta, která se k tomuhle testování používá, ale co mají dělat ti, kteří se nikdy latinsky neučili?

Určitě existuje nějaká pěkná latinská věta, která se k tomuhle testování používá, ale co mají dělat ti, kteří se nikdy latinsky neučili? Určitě existuje nějaká pěkná latinská věta, která se k tomuhle testování používá, ale co mají dělat ti, kteří se nikdy latinsky neučili? Určitě existuje nějaká pěkná latinská věta, která se k tomuhle testování používá, ale co mají dělat ti, kteří se nikdy latinsky neučili? Určitě existuje nějaká pěkná latinská věta, která se k tomuhle testování používá, ale co mají dělat ti, kteří se nikdy latinsky neučili? Určitě existuje nějaká pěkná latinská věta, která se k tomuhle testování používá, ale co mají dělat ti, kteří se nikdy latinsky neučili?

Určitě existuje nějaká pěkná latinská věta, která se k tomuhle testování používá, ale co mají dělat ti, kteří se nikdy latinsky neučili? Určitě existuje nějaká pěkná latinská věta, která se k tomuhle testování používá, ale co mají dělat ti, kteří se nikdy latinsky neučili? Určitě existuje nějaká pěkná latinská věta, která se k tomuhle testování používá, ale co mají dělat ti, kteří se nikdy latinsky neučili? Určitě existuje nějaká pěkná latinská věta, která se k tomuhle testování používá, ale co mají dělat ti, kteří se nikdy latinsky neučili? Určitě existuje nějaká pěkná latinská věta, která se k tomuhle testování používá, ale co mají dělat ti, kteří se nikdy latinsky neučili?

Určitě existuje nějaká pěkná latinská věta, která se k tomuhle testování používá, ale co mají dělat ti, kteří se nikdy latinsky neučili? Určitě existuje nějaká pěkná latinská věta, která se k tomuhle testování používá, ale co mají dělat ti, kteří se nikdy latinsky neučili? Určitě existuje nějaká pěkná latinská věta, která se k tomuhle testování používá, ale co mají dělat ti, kteří se nikdy latinsky neučili? Určitě existuje nějaká pěkná latinská věta, která se k tomuhle testování používá, ale co mají dělat ti, kteří se nikdy latinsky neučili? Určitě existuje nějaká pěkná latinská věta, která se k tomuhle testování používá, ale co mají dělat ti, kteří se nikdy latinsky neučili?

Určitě existuje nějaká pěkná latinská věta, která se k tomuhle testování používá, ale co mají dělat ti, kteří se nikdy latinsky neučili? Určitě existuje nějaká pěkná latinská věta, která se k tomuhle testování používá, ale co mají dělat ti, kteří se nikdy latinsky neučili? Určitě existuje nějaká pěkná latinská věta, která se k tomuhle testování používá, ale co mají dělat ti, kteří se nikdy latinsky neučili? Určitě existuje nějaká pěkná latinská věta, která se k tomuhle testování používá, ale co mají dělat ti, kteří se nikdy latinsky neučili? Určitě existuje nějaká pěkná latinská věta, která se k tomuhle testování používá, ale co mají dělat ti, kteří se nikdy latinsky neučili?

Určitě existuje nějaká pěkná latinská věta, která se k tomuhle testování používá, ale co mají dělat ti, kteří se nikdy latinsky neučili? Určitě existuje nějaká pěkná latinská věta, která se k tomuhle testování používá, ale co mají dělat ti, kteří se nikdy latinsky neučili? Určitě existuje nějaká pěkná latinská věta, která se k tomuhle testování používá, ale co mají dělat ti, kteří se nikdy latinsky neučili? Určitě existuje nějaká pěkná latinská věta, která se k tomuhle testování používá, ale co mají dělat ti, kteří se nikdy latinsky neučili? Určitě existuje nějaká pěkná latinská věta, která se k tomuhle testování používá, ale co mají dělat ti, kteří se nikdy latinsky neučili?

Určitě existuje nějaká pěkná latinská věta, která se k tomuhle testování používá, ale co mají dělat ti, kteří se nikdy latinsky neučili? Určitě existuje nějaká pěkná latinská věta, která se k tomuhle testování používá, ale co mají dělat ti, kteří se nikdy latinsky neučili? Určitě existuje nějaká pěkná latinská věta, která se k tomuhle testování používá, ale co mají dělat ti, kteří se nikdy latinsky neučili? Určitě existuje nějaká pěkná latinská věta, která se k tomuhle testování používá, ale co mají dělat ti, kteří se nikdy latinsky neučili? Určitě existuje nějaká pěkná latinská věta, která se k tomuhle testování používá, ale co mají dělat ti, kteří se nikdy latinsky neučili?

Určitě existuje nějaká pěkná latinská věta, která se k tomuhle testování používá, ale co mají dělat ti, kteří se nikdy latinsky neučili? Určitě existuje nějaká pěkná latinská věta, která se k tomuhle testování používá, ale co mají dělat ti, kteří se nikdy latinsky neučili? Určitě existuje nějaká pěkná latinská věta, která se k tomuhle testování používá, ale co mají dělat ti, kteří se nikdy latinsky neučili? Určitě existuje nějaká pěkná latinská věta, která se k tomuhle testování používá, ale co mají dělat ti, kteří se nikdy latinsky neučili? Určitě existuje nějaká pěkná latinská věta, která se k tomuhle testování používá, ale co mají dělat ti, kteří se nikdy latinsky neučili?

Určitě existuje nějaká pěkná latinská věta, která se k tomuhle testování používá, ale co mají dělat ti, kteří se nikdy latinsky neučili? Určitě existuje nějaká pěkná latinská věta, která se k tomuhle testování používá, ale co mají dělat ti, kteří se nikdy latinsky neučili? Určitě existuje nějaká pěkná latinská věta, která se k tomuhle testování používá, ale co mají dělat ti, kteří se nikdy latinsky neučili? Určitě existuje nějaká pěkná latinská věta, která se k tomuhle testování používá, ale co mají dělat ti, kteří se nikdy latinsky neučili? Určitě existuje nějaká pěkná latinská věta, která se k tomuhle testování používá, ale co mají dělat ti, kteří se nikdy latinsky neučili?

Určitě existuje nějaká pěkná latinská věta, která se k tomuhle testování používá, ale co mají dělat ti, kteří se nikdy latinsky neučili? Určitě existuje nějaká pěkná latinská věta, která se k tomuhle testování používá, ale co mají dělat ti, kteří se nikdy latinsky neučili? Určitě existuje nějaká pěkná latinská věta, která se k tomuhle testování používá, ale co mají dělat ti, kteří se nikdy latinsky neučili? Určitě existuje nějaká pěkná latinská věta, která se k tomuhle testování používá, ale co mají dělat ti, kteří se nikdy latinsky neučili? Určitě existuje nějaká pěkná latinská věta, která se k tomuhle testování používá, ale co mají dělat ti, kteří se nikdy latinsky neučili?

Určitě existuje nějaká pěkná latinská věta, která se k tomuhle testování používá, ale co mají dělat ti, kteří se nikdy latinsky neučili? Určitě existuje nějaká pěkná latinská věta, která se k tomuhle testování používá, ale co mají dělat ti, kteří se nikdy latinsky neučili? Určitě existuje nějaká pěkná latinská věta, která se k tomuhle testování používá, ale co mají dělat ti, kteří se nikdy latinsky neučili? Určitě existuje nějaká pěkná latinská věta, která se k tomuhle testování používá, ale co mají dělat ti, kteří se nikdy latinsky neučili? Určitě existuje nějaká pěkná latinská věta, která se k tomuhle testování používá, ale co mají dělat ti, kteří se nikdy latinsky neučili?

Určitě existuje nějaká pěkná latinská věta, která se k tomuhle testování používá, ale co mají dělat ti, kteří se nikdy latinsky neučili? Určitě existuje nějaká pěkná latinská věta, která se k tomuhle testování používá, ale co mají dělat ti, kteří se nikdy latinsky neučili? Určitě existuje nějaká pěkná latinská věta, která se k tomuhle testování používá, ale co mají dělat ti, kteří se nikdy latinsky neučili? Určitě existuje nějaká pěkná latinská věta, která se k tomuhle testování používá, ale co mají dělat ti, kteří se nikdy latinsky neučili? Určitě existuje nějaká pěkná latinská věta, která se k tomuhle testování používá, ale co mají dělat ti, kteří se nikdy latinsky neučili?

Určitě existuje nějaká pěkná latinská věta, která se k tomuhle testování používá, ale co mají dělat ti, kteří se nikdy latinsky neučili? Určitě existuje nějaká pěkná latinská věta, která se k tomuhle testování používá, ale co mají dělat ti, kteří se nikdy latinsky neučili? Určitě existuje nějaká pěkná latinská věta, která se k tomuhle testování používá, ale co mají dělat ti, kteří se nikdy latinsky neučili? Určitě existuje nějaká pěkná latinská věta, která se k tomuhle testování používá, ale co mají dělat ti, kteří se nikdy latinsky neučili? Určitě existuje nějaká pěkná latinská věta, která se k tomuhle testování používá, ale co mají dělat ti, kteří se nikdy latinsky neučili?

Určitě existuje nějaká pěkná latinská věta, která se k tomuhle testování používá, ale co mají dělat ti, kteří se nikdy latinsky neučili? Určitě existuje nějaká pěkná latinská věta, která se k tomuhle testování používá, ale co mají dělat ti, kteří se nikdy latinsky neučili? Určitě existuje nějaká pěkná latinská věta, která se k tomuhle testování používá, ale co mají dělat ti, kteří se nikdy latinsky neučili? Určitě existuje nějaká pěkná latinská věta, která se k tomuhle testování používá, ale co mají dělat ti, kteří se nikdy latinsky neučili? Určitě existuje nějaká pěkná latinská věta, která se k tomuhle testování používá, ale co mají dělat ti, kteří se nikdy latinsky neučili?

%*****************************************************************************
\chapter{Pokyny a návody k formátování textu práce}
\textbf{\large Tato příloha samozřejmě nebude součástí vaší práce. Slouží pouze jako příklad formátování textu.}

Používat se dají všechny příkazy systému \LaTeX. Existuje velké množství volně přístupné dokumentace, tutoriálů, příruček a dalších materiálů v elektronické podobě. Výchozím bodem, kromě Googlu, může být stránka CSTUG (Czech Tech Users Group) cite{CSTUG}. Tam najdete odkazy na další materiály.  Vetšinou dostačující a přehledně organizovanou elektronikou dokumentaci najdete například na cite{latexdocweb} nebo cite{latexwiki}.

Existují i různé nadstavby nad systémy \TeX{} a \LaTeX, které výrazně usnadní psaní textu zejména začátečníkům. Velmi rozšířený v Linuxovém prostředí je systém Kile.


\section{Vkládání obrázků}
Obrázky se umísťují do plovoucího prostředí \verb|figure|. Každý obrázek by měl obsahovat \textbf{název} (\verb|\caption|) a \textbf{návěští} (\verb|\label|). Použití příkazu pro vložení obrázku \\\verb|\includegraphics| je podmíněno aktivací (načtením) balíku graphicx příkazem\\ \verb|\usepackage{graphicx}|.

Budete-li zdrojový text zpracovávat pomocí programu \verb|pdflatex|, očekávají se obrázky s příponou \verb|*.pdf|\footnote{pdflatex umí také formáty PNG a JPG.}, použijete-li k formátování \verb|latex|, očekávají se obrázky s příponou \verb|*.eps|.\footnote{Vzájemnou konverzi mezi snad všemi typy obrazku včetně změn vekostí a dalších vymožeností vám může zajistit balík ImageMagic  (http://www.imagemagick.org/script/index.php). Je dostupný pod Linuxem, Mac OS i MS Windows. Důležité jsou zejména příkazy convert a identify.}

\begin{figure}[ht]
\begin{center}
\includegraphics[width=5cm]{figures/LogoCVUT}
\caption{Popiska obrázku}
\label{fig:logo}
\end{center}
\end{figure}

Příklad vložení obrázku:
\begin{verbatim}
\begin{figure}[h]
\begin{center}
\includegraphics[width=5cm]{figures/LogoCVUT}
\caption{Popiska obrazku}
\label{fig:logo}
\end{center}
\end{figure}
\end{verbatim}

\section{Kreslení obrázků}
Zřejmě každý z vás má nějaký oblíbený nástroj pro tvorbu obrázků. Jde jen o to, abyste dokázali obrázek uložit v požadovaném formátu nebo jej do něj konvertovat (viz předchozí kapitola). Je zřejmě vhodné kreslit obrázky vektorově. Celkem oblíbený, na ovládání celkem jednoduchý a přitom dostatečně mocný je například program Inkscape.

Zde stojí za to upozornit na kreslící programe Ipe cite{ipe}, který dokáže do obrázku vkládat komentáře přímo v latexovském formátu (vzroce, stejné fonty atd.). Podobné věci umí na Linuxové platformě nástroj Xfig. 

Za pozornost ještě stojí schopnost editoru Ipe importovat obrázek (jpg nebo bitmap) a krelit do něj latexovské popisky a komentáře. Výsledek pak umí exportovat přímo do pdf.

\section{Tabulky}
Existuje více způsobů, jak sázet tabulky. Například je možno použít prostředí \verb|table|, které je velmi podobné prostředí \verb|figure|. 

\begin{table}
\begin{center}
\begin{tabular}{|c|l|l|}
\hline
\textbf{DTD} & \textbf{construction} & \textbf{elimination} \\
\hline
$\mid$ & \verb+in1|A|B a:sum A B+ & \verb+case([_:A]a)([_:B]a)ab:A+\\
&\verb+in1|A|B b:sum A B+ & \verb+case([_:A]b)([_:B]b)ba:B+\\
\hline
$+$&\verb+do_reg:A -> reg A+&\verb+undo_reg:reg A -> A+\\
\hline
$*,?$& the same like $\mid$ and $+$ & the same like $\mid$ and $+$\\
& with \verb+emtpy_el:empty+ & with \verb+emtpy_el:empty+\\
\hline
R(a,b) & \verb+make_R:A->B->R+ & \verb+a: R -> A+\\
 & & \verb+b: R -> B+\\
\hline
\end{tabular}
\end{center}
\caption{Ukázka tabulky}
\label{tab:tab1}
\end{table}

Zdrojový text tabulky \ref{tab:tab1} vypadá takto:
\begin{verbatim}
\begin{table}
\begin{center}
\begin{tabular}{|c|l|l|}
\hline
\textbf{DTD} & \textbf{construction} & \textbf{elimination} \\
\hline
$\mid$ & \verb+in1|A|B a:sum A B+ & \verb+case([_:A]a)([_:B]a)ab:A+\\
&\verb+in1|A|B b:sum A B+ & \verb+case([_:A]b)([_:B]b)ba:B+\\
\hline
$+$&\verb+do_reg:A -> reg A+&\verb+undo_reg:reg A -> A+\\
\hline
$*,?$& the same like $\mid$ and $+$ & the same like $\mid$ and $+$\\
& with \verb+emtpy_el:empty+ & with \verb+emtpy_el:empty+\\
\hline
R(a,b) & \verb+make_R:A->B->R+ & \verb+a: R -> A+\\
 & & \verb+b: R -> B+\\
\hline
\end{tabular}
\end{center}
\caption{Ukázka tabulky}
\label{tab:tab1}
\end{table}
\begin{table}
\end{verbatim}

\section{Odkazy v textu}
\subsection{Odkazy na literaturu}
Jsou realizovány příkazem \verb|\cite{odkaz}|. 

Seznam literatury je dobré zapsat do samostatného souboru a ten pak zpracovat programem bibtex (viz soubor \verb|reference.bib|). Zdrojový soubor pro \verb|bibtex| vypadá například takto:
\begin{verbatim}
@Article{Chen01,
  author  = "Yong-Sheng Chen and Yi-Ping Hung and Chiou-Shann Fuh",
  title   = "Fast Block Matching Algorithm Based on 
             the Winner-Update Strategy",
  journal = "IEEE Transactions On Image Processing",
  pages   = "1212--1222",
  volume  =  10,
  number  =   8,
  year    = 2001,
}

@Misc{latexdocweb,
  author  = "",
  title   = "{\LaTeX} --- online manuál",
  note    = "\verb|http://www.cstug.cz/latex/lm/frames.html|",
  year    = "",
}
...
\end{verbatim}

%11.12.2008, 3.5.2009
\textbf{Pozor:} Sazba názvů odkazů je dána Bib\TeX{} stylem\\ (\verb|\bibliographystyle{abbrv}|). 
%Budete-li používat české prostředí (\verb|\usepackage[czech]{babel}|), 
Bib\TeX{} tedy obvykle vysází velké pouze počáteční písmeno z názvu zdroje, 
ostatní písmena zůstanou malá bez ohledu na to, jak je napíšete. 
Přesněji řečeno, styl může zvolit pro každý typ publikace jiné konverze. 
Pro časopisecké články třeba výše uvedené, jiné pro monografie (u nich často bývá 
naopak velikost písmen zachována).

Pokud chcete Bib\TeX u napovědět, která písmena nechat bez konverzí 
(viz \texttt{title = "\{$\backslash$LaTeX\} -{}-{}- online manuál"} 
v~předchozím příkladu), je nutné příslušné písmeno (zde celé makro) uzavřít 
do složených závorek. Pro přehlednost je proto vhodné celé parametry 
uzavírat do uvozovek (\texttt{author = "\dots"}), nikoliv do složených závorek.

Odkazy na literaturu ve zdrojovém textu se pak zapisují:
\begin{verbatim}
Podívejte se na cite{Chen01}, 
další detaily najdete na cite{latexdocweb}
\end{verbatim}

Vazbu mezi soubory \verb|*.tex| a \verb|*.bib| zajistíte příkazem 
\verb|\bibliography{}| v souboru \verb|*.tex|.  V našem případě tedy zdrojový 
dokument \verb|thesis.tex| obsahuje příkaz\\
\verb|\bibliography{reference}|.

Zpracování zdrojového textu s odkazy se provede postupným voláním programů\\
\verb|pdflatex <soubor>| (případně \verb|latex <soubor>|), \verb|bibtex <soubor>| 
a opět\\ \verb|pdflatex <soubor>|.\footnote{První volání \texttt{pdflatex} 
vytvoří soubor s~koncovkou \texttt{*.aux}, který je vstupem pro program 
\texttt{bibtex}, pak je potřeba znovu zavolat program \texttt{pdflatex} 
(\texttt{latex}), který tentokrát zpracuje soubory s příponami \texttt{.aux} a 
\texttt{.tex}. 
Informaci o případných nevyřešených odkazech (cross-reference) vidíte přímo při 
zpracovávání zdrojového souboru příkazem \texttt{pdflatex}. Program \texttt{pdflatex} 
(\texttt{latex}) lze volat vícekrát, pokud stále vidíte nevyřešené závislosti.}


Níže uvedený příklad je převzat z dříve existujících pokynů studentům, kteří 
dělají svou diplomovou nebo bakalářskou práci v~Grafické skupině.\footnote{Několikrát 
jsem byl upozorněn, že web s těmito pokyny byl zrušen, proto jej zde přímo necituji. 
Nicméně příklad sám o sobě dokumentuje obecně přijímaný konsensus ohledně citací 
v~bakalářských a diplomových pracích na KP.} Zde se praví:
\begin{small}
\begin{verbatim}
...
j) Seznam literatury a dalších použitých pramenů, odkazy na WWW stránky, ...
 Pozor na to, že na veškeré uvedené prameny se musíte v textu práce 
 odkazovat -- [1]. 
Pramen, na který neodkazujete, vypadá, že jste ho vlastně nepotřebovali 
a je uveden jen do počtu. Příklad citace knihy [1], článku v časopise [2], 
stati ve sborníku [3] a html odkazu [4]: 
[1] J. Žára, B. Beneš;, and P. Felkel. 
     Moderní počítačová grafika. Computer Press s.r.o, Brno, 1 edition, 1998. 
     (in Czech). 
[2] P. Slavík. Grammars and Rewriting Systems as Models for Graphical User 
     Interfaces. Cognitive Systems, 4(4--3):381--399, 1997. 
[3] M. Haindl, Š. Kment, and P. Slavík. Virtual Information Systems. 
     In WSCG'2000 -- Short communication papers, pages 22--27, Pilsen, 2000. 
     University of West Bohemia. 
[4] Knihovna grafické skupiny katedry počítačů: 
     http://www.cgg.cvut.cz/Bib/library/ 
\end{verbatim}
\end{small}
\ldots{} abychom výše citované odkazy skutečně našli v (automaticky generovaném) seznamu literatury tohoto textu, musíme je nyní alespoň jednou citovat: Kniha cite{kniha}, článek v~časopisu cite{clanek}, příspěvek na konferenci {sbornik}, www odkaz cite{www}.

Ještě přidáme další ukázku citací online zdrojů podle české normy. Odkaz na wiki o frameworcich  a ORM cite{wiki:orm}. Použití viz soubor \verb|reference.bib|. V seznamu literatury by nyní měly být živé odkazy na zdroje. V \verb|reference.bib| je zcela nový typ publikace. Detaily dohledal a dodal Petr Dlouhý v dubnu 2010. Podrobnosti najdete ve zdrojovém souboru tohoto textu v komentáři u příkazu \verb|\thebibliography|.

\subsection{Odkazy na obrázky, tabulky a kapitoly}
\begin{itemize}
\item Označení místa v textu, na které chcete později čtenáře práce odkázat, se provede příkazem \verb|\label{navesti}|. Lze použít v prostředích \verb|figure| a  \verb|table|, ale též za názvem kapitoly nebo podkapitoly.
\item Na návěští se odkážeme příkazem \verb|\ref{navesti}| nebo \verb|\pageref{navesti}|.
\end{itemize}

\section{Rovnice, centrovaná, číslovaná matematika}
Jednoduchý matematický výraz zapsaný přímo do textu se vysází pomocí prostředí \verb|math|, resp. zkrácený zápis pomocí uzavření textu rovnice mezi znaky \verb|$|.

Kód \verb|$ S = \pi * r^2 $| bude vysázen takto: $ S = \pi * r^2 $.

Pokud chcete nečíslované rovnice, ale umístěné centrovaně na samostatné řádky, pak lze použít prostředí \verb|displaymath|, resp. zkrácený zápis pomocí uzavření textu rovnice mezi znaky \verb|$$|. Zdrojový kód: 
\begin{verb}
|$$ S = \pi * r^2 $$|
\end{verb}
bude pak vysázen takto:
$$ S = \pi * r^2 $$

Chcete-li mít rovnice číslované, je třeba použít prostředí \verb|eqation|. Kód:
\begin{verbatim}
\begin{equation}
  S = \pi * r^2
\end{equation}

\begin{equation}
  V = \pi * r^3
\end{equation}
\end{verbatim}
je potom vysázen takto:
\begin{equation}
  S = \pi * r^2
\end{equation}

\begin{equation}
  V = \pi * r^3
\end{equation}

\section{Kódy programu}
Chceme-li vysázet například část zdrojového kódu programu (bez formátování), hodí se prostředí \verb|verbatim|: 
\begin{verbatim}
         (* nickname2 *)
Lego> Refine in1
             (do_reg (nickname1 h));
Refine by  in1 (do_reg (nickname1 h))
   ?4 : pcdata
   ?5 : pcdata
          (* surname2 *)
Lego> Refine surname1 h;
Refine by  surname1 h
   ?5 : pcdata
          (* email2 *)
Lego> Refine undo_reg (email1 h);
Refine by  undo_reg (email1 h)
*** QED ***
\end{verbatim}

\section{Další poznámky}
\subsection{České uvozovky}
V souboru \verb|k336_thesis_macros.tex| je příkaz \verb|\uv{}| pro sázení českých uvozovek. \uv{Text uzavřený do českých uvozovek.}

% JZ: 3.5.2009 \chapter z book zajistí automaticky
%\subsection{Začátky kapitol na liché stránky}
%Ve výsledném textu je dobré, když každá kapitola začíná na liché stránce. Tedy použijte:
%\begin{verbatim}
%  \cleardoublepage\include{1_uvod}
%  \cleardoublepage\include{2_teorie}
%   atd.\ldots{}
%\end{verbatim}

%*****************************************************************************
\chapter{Seznam použitých zkratek}

\begin{description}
\item[2D] Two-Dimensional
\item[ABN] Abstract Boolean Networks
\item[ASIC] Application-Specific Integrated Circuit
\end{description}
\vdots

%*****************************************************************************
\chapter{UML diagramy}
\textbf{\large Tato příloha není povinná a zřejmě se neobjeví v každé práci. Máte-li ale větší množství podobných diagramů popisujících systém, není nutné všechny umísťovat do hlavního textu, zvláště pokud by to snižovalo jeho čitelnost.}

%*****************************************************************************
\chapter{Instalační a uživatelská příručka}
\textbf{\large Tato příloha velmi žádoucí zejména u softwarových implementačních prací.}

%*****************************************************************************
\chapter{Obsah přiloženého CD}
\textbf{\large Tato příloha je povinná pro každou práci. Každá práce musí totiž obsahovat přiložené CD. Viz dále.}

Může vypadat například takto. Váš seznam samozřejmě bude odpovídat typu vaší práce. (viz cite{infodp}):

\begin{figure}[h]
\begin{center}
%\includegraphics[width=14cm]{figures/seznamcd}
\caption{Seznam přiloženého CD --- příklad}
%\label{fig:seznamcd}
\end{center}
\end{figure}

Na GNU/Linuxu si strukturu přiloženého CD můžete snadno vyrobit příkazem:\\ 
\verb|$ tree . >tree.txt|\\
Ve vzniklém souboru pak stačí pouze doplnit komentáře.

Z \textbf{README.TXT} (případne index.html apod.)  musí být rovněž zřejmé, jak programy instalovat, spouštět a jaké požadavky mají tyto programy na hardware.

Adresář \textbf{text}  musí obsahovat soubor s vlastním textem práce v PDF nebo PS formátu, který bude později použit pro prezentaci diplomové práce na WWW.

\end{document}
